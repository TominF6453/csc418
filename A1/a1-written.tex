\documentclass[12pt]{article}
\usepackage{fullpage,amsmath,amssymb,amsthm,etoolbox}
\usepackage{tikz}
\usepackage{hyperref}
\hypersetup{pdfpagemode=FullScreen}
\usepackage{mathtools}
\usepackage{qtree}
\DeclarePairedDelimiter{\ceil}{\lceil}{\rceil}
\DeclarePairedDelimiter{\floor}{\lfloor}{\rfloor}
\DeclarePairedDelimiter{\abs}{\lvert}{\rvert}
\DeclarePairedDelimiter{\norm}{\lvert\lvert}{\rvert\rvert}

\title{CSC418 Assignment \#1\\}
\author{Filip Tomin\\
\normalsize{1001329984}}

\date{}
\begin{document}
\maketitle
\begin{enumerate}
%Question 1
\item
\begin{enumerate}
\item %IMPLICIT FORM
To compute the implicit form, we'll rearrange $x(t)$ for $t$ and then plug that into the equation for $y$.\\
$x = 2sin(t)\rightarrow t = arcsin(\frac{x}{2})\rightarrow y = 5sin(arcsin(\frac{x}{2}))cos(arcsin(\frac{x}{2}))$\\
Then we can simply subtract $y$ from both sides to obtain the implicit form:\begin{center}
$0 = 5sin(arcsin(\frac{x}{2}))cos(arcsin(\frac{x}{2})) - y$
\end{center}
\item %TANGENT AND NORMAL VECTOR
The tangent vector can be found by defining the curve as $p(t) = (2sint, 5sintcost)$ and then finding $p'(t)$.\\
$x'(t) = 2cost, y'(t) = 5cos(2t) \rightarrow p'(t) = (2cost, 5cos(2t))$\\
The normal vector is the tangent vector with the elements inverted as such:\\
$p_n(t) = (-5cos(2t), 2cost)$
\item %SYMMETRY
The equation is symmetric about the $x$ axis when $-y(t) = y(-t)$ and vice versa.\\
Note: $sin(-\theta ) = -sin\theta$ and $cos(-\theta ) = cos(\theta )$\\
$y(-t) = 5sin(-t)cos(-t) = 5(-sint)cost = -5sintcost = -y(t)$\\
$x(-t) = 2sin(-t) = 2(-sint) = -2sint = -x(t)$\\
Thus, the equation is symmetric in both the $x$ and $y$ axes.
\item %ENCLOSED AREA OF BOWTIE
To find the area, we need to solve $\int ydx$. Since the bowtie is symmetric, we can integrate from 0 to $\frac{\pi}{2}$ and multiply the result by 4. $y$ is already given and $dx = 2cost$ from the tangent equations.\begin{center}
$4\int_0^\frac{\pi}{2} (5sintcost)(2cost)dt \rightarrow 40\int_0^\frac{\pi}{2} sintcos^2tdt = 40(\frac{1}{3}) = \frac{40}{3}\approx 13.3$
\end{center}
\item %APPROXIMATE PERIMETER OF BOWTIE
Approximating the perimeter is the same as finding the arc length, and since the bowtie is both $x$ and $y$ symmetric, we only need to integrate from 0 to $\frac{\pi}{2}$ and multiply the result by 4. The formula for finding arc length is $L = \int_a^b \sqrt{\frac{dx}{dt}^2 + \frac{dy}{dt}^2}dt$. We already know $dx$ and $dy$ from a previous question:\begin{center}
$4\int_0^\frac{\pi}{2} \sqrt{(2cost)^2 + (5cos(2t))^2}dt = 4\int_0^\frac{\pi}{2} \sqrt{4cos^2t + 25cos^2(2t)}dt\approx 4(5.62)\approx 22.48 $
\end{center}
\end{enumerate}
%Question 2
\item
Two transformations commute if $A\cdot B = B\cdot A$, so that's what we'll check.
\begin{enumerate}
\item %TRANSLATE + TRANSLATE
Translation + translation:\\
$\begin{bmatrix}
1 & 0 & a_1\\
0 & 1 & a_2\\
0 & 0 & 1
\end{bmatrix}\cdot \begin{bmatrix}
1 & 0 & b_1\\
0 & 1 & b_2\\
0 & 0 & 1
\end{bmatrix} = \begin{bmatrix}
1 & 0 & a_1 + b_1\\
0 & 1 & a_2 + b_2\\
0 & 0 & 1
\end{bmatrix}$\\
$\begin{bmatrix}
1 & 0 & b_1\\
0 & 1 & b_2\\
0 & 0 & 1
\end{bmatrix}\cdot \begin{bmatrix}
1 & 0 & a_1\\
0 & 1 & a_2\\
0 & 0 & 1
\end{bmatrix} = \begin{bmatrix}
1 & 0 & b_1 + a_1\\
0 & 1 & b_2 + a_2\\
0 & 0 & 1
\end{bmatrix}$\\ Thus they commute.
\item %TRANSLATE + ROTATE
Translation + rotation:\\
$\begin{bmatrix}
1 & 0 & a_1\\
0 & 1 & a_2\\
0 & 0 & 1
\end{bmatrix}\cdot \begin{bmatrix}
cosb & sinb & 0\\
-sinb & cosb & 0\\
0 & 0 & 1
\end{bmatrix} = \begin{bmatrix}
cosb & sinb & a_1\\
-sinb & cosb & a_2\\
0 & 0 & 1
\end{bmatrix}$\\
$\begin{bmatrix}
cosb & sinb & 0\\
-sinb & cosb & 0\\
0 & 0 & 1
\end{bmatrix}\cdot \begin{bmatrix}
1 & 0 & a_1\\
0 & 1 & a_2\\
0 & 0 & 1
\end{bmatrix} = \begin{bmatrix}
cosb & sinb & a_1cosb + a_2sinb\\
-sinb & cosb & -a_1sinb + a_2cosb\\
0 & 0 & 1
\end{bmatrix}$\\ Thus they do not commute.
\item %SCALING + ROTATION, DIFF POINTS
These do not commute, but it is easier to show via an example.\\
We start with a square, $p_1, p_2, p_3, p_4 = (0,0),(1,0),(0,1),(1,1)$. Our scaling transform with simply be a $\times 2$ scale with fixed point $p_1$, and our rotation transform will be a 90 degree clockwise rotation around fixed point $p_2$.\\
Scaling then rotation:\begin{center}
$\begin{bmatrix}
(0,0)\\(1,0)\\(0,1)\\(1,1)
\end{bmatrix}\rightarrow \begin{bmatrix}
(0,0)\\(2,0)\\(0,2)\\(2,2)
\end{bmatrix}\rightarrow \begin{bmatrix}
(2,2)\\(2,0)\\(4,2)\\(4,0)
\end{bmatrix}$
\end{center} Rotating then scaling:\begin{center}
$\begin{bmatrix}
(0,0)\\(1,0)\\(0,1)\\(1,1)
\end{bmatrix}\rightarrow \begin{bmatrix}
(1,1)\\(1,0)\\(2,1)\\(2,0)
\end{bmatrix}\rightarrow \begin{bmatrix}
(1,1)\\(1,-1)\\(3,1)\\(3,-1)
\end{bmatrix}$
\end{center} Thus, they do not commute.
\item %SCALING + SCALING, SAME POINTS
Scaling + scaling:\\
$\begin{bmatrix}
a_1 & 0 & 0\\
0 & a_2 & 0\\
0 & 0 & 1
\end{bmatrix}\cdot \begin{bmatrix}
b_1 & 0 & 0\\
0 & b_2 & 0\\
0 & 0 & 1
\end{bmatrix} = \begin{bmatrix}
a_1b_1 & 0 & 0\\
0 & a_2b_2 & 0\\
0 & 0 & 1
\end{bmatrix}$\\
$\begin{bmatrix}
b_1 & 0 & 0\\
0 & b_2 & 0\\
0 & 0 & 1
\end{bmatrix}\cdot \begin{bmatrix}
a_1 & 0 & 0\\
0 & a_2 & 0\\
0 & 0 & 1
\end{bmatrix} = \begin{bmatrix}
b_1a_1 & 0 & 0\\
0 & b_2a_2 & 0\\
0 & 0 & 1
\end{bmatrix}$\\ Thus they commute. When scaling with a fixed point, all the other points are locked into the line they form with the fixed point, scaling the point only scales the distance between it and the fixed point along that line. This is equivalent to the fixed point being the origin, thus the above matrix proof applies to any fixed point scale, as long as the fixed point is the same for all scaling transforms.
\end{enumerate}
%Question 3
\item %AFFINE TRANSFORM
First, noticing that $x$ values of 0 map to 7 and $x$ values of 1 map to 6, it seems like the square formed by these points flips horizontally. To achieve this, we first need to scale the $x$-axis by -1:\begin{center}
$T_1 = \begin{bmatrix}
-1 & 0 & 0\\
0 & 1 & 0\\
0 & 0 & 1
\end{bmatrix}$
\end{center}
Then, by subtracting the mapped points to the new points, we find every point translates by $(7,2)$, so we apply a simple translation:\begin{center}
$T_2 = \begin{bmatrix}
1 & 0 & 7\\
0 & 1 & 2\\
0 & 0 & 1
\end{bmatrix} \rightarrow T_2T_1 = \begin{bmatrix}
-1 & 0 & 7\\
0 & 1 & 2\\
0 & 0 & 1
\end{bmatrix}$
\end{center}
Applying these two transforms in that order give the correct mapping.\\
The point $(2,5)$ under these transforms would get mapped to $(5,7)$.
%Question 4
\item %TRIANGLE
First, we calculate the lines between all of the vertices, so $v_0v_1, v_1v_2$ and $v_2v_0$. Then we calculate the area of the triangle by these lines, obtaining $v_0v_1v_2$. We then take point $p$ and calculate new triangles substituting one vertex for $p$ each time: $v_0v_1p, v_0pv_2$ and $pv_1v_2$. If the sum of the areas of all these $p$ triangles equals the area of the full triangle, it lies within the triangle.\\
To find the area of the triangle, we can take $v_2$ and subtract it from $v_0$ and $v_1$ in order to translate the triangle to the origin. Then, assuming $v_0 = (a,b)$ and $v_1 = (c,d)$, the area of the triangle is $A = \frac{\abs{ad-bc}}{2}$. Any point can be subtracted, as long as one point is $(0,0)$ and the other two are $(a,b)$ and $(c,d)$.\\
To find the centroid, $q = (x', y')$, we simply take the vertices' x and y values, sum them and divide by 3. So, $x' = \frac{v_{x0} + v_{x1} + v{x2}}{3}$, and similar for $y'$.

\end{enumerate}
\end{document}